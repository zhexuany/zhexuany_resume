%-------------------------------------------------------------------------------
%	SECTION TITLE
%-------------------------------------------------------------------------------
\cvsection{Experience}


%-------------------------------------------------------------------------------
%	CONTENT
%-------------------------------------------------------------------------------
\begin{cventries}
%---------------------------------------------------------
\cventry
{运营合伙人兼客户工程团队 VP} % Job title
{Tapdata} % Organization
{上海, 中国} % Location
{Mar. 2022 - Sept. 2022} % Date(s)
{
  \begin{cvitems} % Description(s) of tasks/responsibilities
    \item {售前售后团队, 销售赋能体系,产品市场定位等工作}
  \end{cvitems}
}

%---------------------------------------------------------
  \cventry
    {咨询 \& 华南区售前总监} % Job title
    {中国商业化事业部, PingCAP} % Organization
    {深圳, 广东} % Location
    {Aug. 2020 - Jan. 2022} % Date(s)
    {
      \begin{cvitems} % Description(s) of tasks/responsibilities
        \item {从 0 到 1 搭建专业售前团队,对售前能力进行抽象撰写培训材料,专注让团队更有效率和骄傲的工作。}
        \item {根据区域客户画像和细分市场特点制定销售策略完成区域销售目标。}
        \item {区域内优化和迭代售前作为资源方和销售配合的最佳实践, 提升人效和资源使用效率。}
      	\item {任期内发起、加速了,加深了 TiDB 作为一个技术在华南金融行业(招商银行、广发银行、广发证券、安信证券等)的使用。}
      \end{cvitems}
    }

%---------------------------------------------------------
  \cventry
    {咨询 \& 资深解决方案顾问 } % Job title
    {中国商业化事业部, PingCAP} % Organization
    {上海, 中国} % Location
    {Dec. 2019 - Aug. 2020} % Date(s)
    {
      \begin{cvitems} % Description(s) of tasks/responsibilities
        \item {以 TiDB 解决方案专家的身份参与到华东地区金融客户的打单过程,通过与客户/利益相关者的一系列研讨会收集业务/技术要求,分析和调查可能的解决方案,以满足客户的业务和技术要求。}
      \end{cvitems}
    }

%---------------------------------------------------------
  \cventry
    {数据库内核工程师 \enskip|\enskip 技术栈: Java, Scala, Golang, Transaction, Optimizator, Spark, TiDB} % Job title
    {研发事业部, PingCAP} % Organization
    {上海, 中国} % Location
    {Jan. 2017 - Dec. 2019} % Date(s)
    {
      \begin{cvitems} % Description(s) of tasks/responsibilities
        \item {和团队一起从 0 开始,构建一个可以将 TiDB 生态和 Spark 生态链接起来的项目,TiSpark。}
        \item {设计并使 TiSpark 能够直接插入到底层存储引擎 TiKV 中,使得写入可以和 TiDB 解耦,让用户多了一种更快的写入选择。}
        \item {专注于 SQL 解析器,优化器,执行和可用性的东西,提升 TiDB 作为一个商业产品的可观测性。}
        \item {对 Raft、Transaction、LSM-Tree 有深刻的理解。}
        \item {TiSpark 项目核心作者,github.com/pingcap/tispark, Github 贡献度排名第 3。}
        \item {TiDB 项目核心作者,github.com/pingcap/tidb, Github 贡献度排名第 40。}
      \end{cvitems}
    }
%---------------------------------------------------------
\end{cventries}
